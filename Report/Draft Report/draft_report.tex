%%%%%%%%%%%%%%%%%%%%%%%%%%%%%%%%%%%%%%%%%
% Short Sectioned Assignment
% LaTeX Template
% Version 1.0 (5/5/12)
%
% This template has been downloaded from:
% http://www.LaTeXTemplates.com
%
% Original author:
% Frits Wenneker (http://www.howtotex.com)
%
% License:
% CC BY-NC-SA 3.0 (http://creativecommons.org/licenses/by-nc-sa/3.0/)
%
%%%%%%%%%%%%%%%%%%%%%%%%%%%%%%%%%%%%%%%%%

%----------------------------------------------------------------------------------------
%	PACKAGES AND OTHER DOCUMENT CONFIGURATIONS
%----------------------------------------------------------------------------------------

\documentclass[paper=a4, fontsize=11pt]{scrartcl} % A4 paper and 11pt font size

\usepackage[T1]{fontenc} % Use 8-bit encoding that has 256 glyphs
\usepackage{fourier} % Use the Adobe Utopia font for the document - comment this line to return to the LaTeX default
\usepackage[english]{babel} % English language/hyphenation
\usepackage{amsmath,amsfonts,amsthm} % Math packages

\usepackage{lipsum} % Used for inserting dummy 'Lorem ipsum' text into the template

\usepackage{sectsty} % Allows customizing section commands
\allsectionsfont{\centering \normalfont\scshape} % Make all sections centered, the default font and small caps

\usepackage{fancyhdr} % Custom headers and footers
\pagestyle{fancyplain} % Makes all pages in the document conform to the custom headers and footers
\fancyhead{} % No page header - if you want one, create it in the same way as the footers below
\fancyfoot[L]{} % Empty left footer
\fancyfoot[C]{} % Empty center footer
\fancyfoot[R]{\thepage} % Page numbering for right footer
\renewcommand{\headrulewidth}{0pt} % Remove header underlines
\renewcommand{\footrulewidth}{0pt} % Remove footer underlines
\setlength{\headheight}{13.6pt} % Customize the height of the header

\numberwithin{equation}{section} % Number equations within sections (i.e. 1.1, 1.2, 2.1, 2.2 instead of 1, 2, 3, 4)
\numberwithin{figure}{section} % Number figures within sections (i.e. 1.1, 1.2, 2.1, 2.2 instead of 1, 2, 3, 4)
\numberwithin{table}{section} % Number tables within sections (i.e. 1.1, 1.2, 2.1, 2.2 instead of 1, 2, 3, 4)

\setlength\parindent{0pt} % Removes all indentation from paragraphs - comment this line for an assignment with lots of text

%----------------------------------------------------------------------------------------
%	TITLE SECTION
%----------------------------------------------------------------------------------------

\newcommand{\horrule}[1]{\rule{\linewidth}{#1}} % Create horizontal rule command with 1 argument of height

\title{	
\normalfont \normalsize 
\textsc{International Institute of Information Technology, Bangalore} \\ [25pt] % Your university, school and/or department name(s)
\horrule{0.5pt} \\[0.4cm] % Thin top horizontal rule
\huge Self Healing Water Networks \\ % The assignment title
\horrule{2pt} \\[0.5cm] % Thick bottom horizontal rule
}

\author{Kumudini Kakwani 
\and Arjun S Bharadwaj
\and Abhijith Madhav}

\date{\normalsize\today} % Today's date or a custom date

\begin{document}

\maketitle % Print the title

%----------------------------------------------------------------------------------------
%	PROBLEM 1
%----------------------------------------------------------------------------------------

\section{Problem Statement}



In the communications world, Self-healing networks are those that are architected in a manner that they can withstand a failure in their transmission paths.
 
Can we apply those concepts to the "water networks"?
 
You are to design an app that helps in this "self-healing" process by collating real-time data from sensors monitoring water flow and water quality to provide analysis of water resources and usage in an area, along with support for decisions on managing water sources.


%------------------------------------------------

\section{Proposal}
\begin{itemize}
\item Collect data on purity of water. 
\item Do time vs purity, water pressure vs time analysis.
\item Analyse water consumption trends with time of the day.
\item Collect data when water is got through tankers. Analyse it.
\item Analyse usage trends across buildings (Academic, hostels, cafeteria) at various granularity levels. Activities include watering plants, cleaning utensils, washrooms and toilets, refilling lake, cleaning, cooking, drinking.
\item Analyse whether using recycled water for gardening etc. will solve the water issue.
\item Controlling of actuators through authorized mobile devices.
\item Notifications to the concerned operational personnel regarding events like 
\begin{itemize}
\item Abnormal chemical levels in water.
\item Change in water table level.
\item Regarding running taps and leakages.
\end{itemize}
\end{itemize}
%----------------------------------------------------------------------------------------

\subsection{Analytics}
Followint reports will be generated.
\begin{itemize}
\item Get the data of number of students in campus and plot the water consumption vs no of students in campus.
\item Analyse the peaks and troughs in the usage of water.
\item Get weather data from data.gov.in or Google and compare it with water usage in the campus.
\item Get data of how long the motor is kept on. Get the electricity usage of the motors. Analyse usage of electricity with hardness of water.
\item Analyse monthly water bill.
\end{itemize}

%------------------------------------------------

\section{Identified Problems}
\begin{itemize}
\item Lack of clarity on number and location of sensors.
\item Integration points are not clear.
\item Scope and requirements are unclear.
\end{itemize}


\end{document}